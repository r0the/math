% ------------------------------------------------------------------------------
% LaTeX-Grundkonfiguration von Stefan Rothe
% ------------------------------------------------------------------------------

% 1 Konfiguration der Schriftarten
% --------------------------------
% um \ifXeTeX verwenden zu können
\usepackage{iftex}

\ifXeTeX
  % Setze PDF-Version auf 1.7
  \special{pdf:minorversion 7}

  % 1.1 Konfiguration der Schriftarten für XeTeX (empfohlen)
  % ----------------------------------------------------------
  \usepackage{fontspec}

  \IfFontExistsTF{Helvetica}{\setmainfont{Helvetica}}{%
    \IfFontExistsTF{Arial}{\setmainfont{Arial}}{}%
  }

  \IfFontExistsTF{Helvetica}{\setsansfont{Helvetica}}{%
    \IfFontExistsTF{Arial}{\setsansfont{Arial}}{}%
  }

  \IfFontExistsTF{Menlo}{\setmonofont[SizeFeatures={Size=10}]{Menlo}}{%
    \IfFontExistsTF{Courier New}{\setmonofont[SizeFeatures={Size=10}]{Courier New}}{}%
  }
\else
  % Setze PDF-Version auf 1.7
  \pdfminorversion=7

  % 1.2 Konfiguration der Schriftarten für PdfTeX
  % -----------------------------------------------

  % Unterstützung von UTF-8 (Unicode)
  \usepackage[utf8]{inputenc}

  % Unterstützung der modernen Zeichencodierung
  \usepackage[T1]{fontenc}

  % Moderne Schriftart verwenden
  \usepackage{lmodern}

  % Schriftart Helvetica verwenden
  \usepackage{helvet}

  % serifenfreie Schriftvariante verwenden
  \renewcommand{\familydefault}{\sfdefault}
\fi

% 2 wichtige Pakete laden und konfigurieren
% -----------------------------------------

% sprachspezifische Anpassungen
\usepackage[ngerman]{babel}

% Absatzabstände kontrollieren für nicht-KOMA-Klassen
\makeatletter
\@ifclassloaded{scrartcl}{}{\usepackage{parskip}}
\makeatother

% Aufzählungen
\usepackage{enumitem}

% mehrere Spalten
\usepackage{multicol}

% Rahmen
\usepackage{mdframed}

% Tabellen
\usepackage{booktabs}
\usepackage{tabularx}

% Grafiken (JPG, PNG, PDF)
\usepackage{graphicx}

% Links
\usepackage{hyperref}
\hypersetup{colorlinks=true,urlcolor=blue,linkcolor=black}

% Einbinden von Quellcode
\usepackage{listings}
\lstdefinestyle{mystyle}{
  basicstyle=\ttfamily,
  numberstyle=\scriptsize\color{black!70},
  numbers=left,
  numbersep=0.75cm
}
\lstset{style=mystyle}

% 3 Mathematik
% ------------

% 3.1 Zahlen und Einheiten schön darstellen
% -----------------------------------------
\usepackage{siunitx}
\sisetup{%
  mode=match,
  exponent-product=\cdot,
  group-digits=integer,
  group-separator={\text{\textquotesingle}}
}

% 3.2 AMS-Mathematik
% ------------------

\usepackage[fleqn]{amsmath}
\usepackage{amssymb}

% 3.3 Vektoren
% ------------
\usepackage[b]{esvect}

\makeatletter
% Koordinatenschreibweise für 2D-Punkte
\NewDocumentCommand\pxy@internal{mmm}{\mathopen{}\left(#1/#2\right)\mathclose{}}
\NewDocumentCommand\pxy{o>{\SplitArgument{2}{,}}m}{\IfNoValueF{#1}{#1}\pxy@internal #2}

% Komponentenschreibweise für 2D-Vektoren
\NewDocumentCommand\vxy@internal{mmm}{\begin{pmatrix}#1\\#2\end{pmatrix}}
\NewDocumentCommand\vxy{>{\SplitArgument{2}{,}}m}{\vxy@internal #1}
% Komponentenschreibweise für 3D-Vektoren
\NewDocumentCommand\vxyz{mmm}{\begin{pmatrix}#1\\#2\\#3\end{pmatrix}}
\makeatother

% Lineare Gleichungssysteme
\usepackage{systeme}
\sysdelim||

% Kürzen bei Brüchen zeigen
\usepackage{cancel}

% eigene Operatoren
\DeclareMathOperator{\ggT}{ggT}
\DeclareMathOperator{\kgV}{kgV}
\DeclareMathOperator{\lb}{lb}

% muss wegen der Konfiguration vor tikz eingebunden werden
% Mit table können Tabellenzellen eingefärbt werden
\usepackage[table]{xcolor}
% für Geometrie (TikZ-Erweiterung)
% damit wird auch tikz eingebunden
\usepackage{tkz-base}
\usepackage{tkz-fct}
% Konfiguration Darstellung von Tangenten in tkz-fct
\tkzfctset{tan style/.style={red,thick,>=}}
\usepackage{tkz-euclide}
% Bäume mit tikz
\usepackage{forest}

% Material Design-Farben
% ----------------------
\definecolor{lightblue}{HTML}{BBDEFB} % MD Blue 100
\definecolor{lightred}{HTML}{FFCDD2} % MD Red 100
\definecolor{lightgrey}{HTML}{F5F5F5} % MD Gray 100
\definecolor{lightgreen}{HTML}{C8E6C9} % MD Green 100
\definecolor{lightorange}{HTML}{FFE0B2} % MD Orange 100

\definecolor{theoremcolor}{HTML}{FFEBEE} % MD Red 50
\definecolor{notecolor}{HTML}{FFECB3} % MD Amber 100

\definecolor{red}{HTML}{D50000} % MD Red A700
\definecolor{green}{HTML}{31843F} % MD Green A700
\definecolor{blue}{HTML}{2962FF} % MD Blue A700
\definecolor{teal}{HTML}{00BFA5} % MD Teal A700
\definecolor{cyan}{HTML}{00B8D4} % MD Cyan A700
\definecolor{yellow}{HTML}{EE8E0D} % WordPress Colors Yellow Fire 40%

\tikzset{dim style/.append style={purple,dashed}}
\tikzset{dim fence style/.append style={purple}}
\tikzset{mark angle style/.append style={german}}

% eigene Befehle
% --------------

\tikzset{circled/.style={shape=circle,draw,inner sep=2pt}}
\NewDocumentCommand\circled{m}{\tikz[baseline=(char.base)]{\node[circled] (char) {#1};}}

% Umgebung eqt
\NewDocumentEnvironment{eqt}{}{\begin{array}{>{\displaystyle}r@{\hspace{0.2cm}}>{\displaystyle}l@{\hspace{1cm}}|l}}{\end{array}}

% Makro \result, um das Resultat von Berechnungen hervorzuheben.
\NewDocumentCommand\result{m}{\textcolor{red}{#1}}

% Makro \extra, um schwierige Zusatzaufgaben zu markieren
\NewDocumentCommand\extra{}{$\bigstar\quad$}

\newmdenv[,skipabove=\parskip,outerlinewidth=1.5pt]{example}
\newmdenv[backgroundcolor=notecolor,skipabove=\parskip,outerlinewidth=1.5pt]{note}
\newmdenv[backgroundcolor=theoremcolor,skipabove=\parskip,outerlinewidth=1.5pt]{theorem}
\newmdenv[backgroundcolor=theoremcolor,outerlinewidth=1.5pt]{instructions}

\usepackage{fontawesome5}
\def\digital{\faLaptop{} }
\def\present{\faTrophy{} }

\def\rosconfig{1}
