\newpage
\section{Kollinearität und Orthogonalität}

Eine wichtige Frage ist es, ob zwei Vektoren parallel zueinander sind oder senkrecht aufeinander stehen.

\subsection{Kollinearität (parallel)}

Sind zwei Vektoren parallel, so werden sie als kollinear bezeichnet. Zwei Vektoren $\vv{a}$ und $\vv{b}$ sind parallel, wenn Sie in die gleiche oder umgekehrte Richtung zeigen. Das ist genau dann der Fall, wenn der eine Vektor ein vielfaches des anderen Vektors ist.

\begin{theorem}
  \textbf{Kollinearität.} Zwei Vektoren $\vv{a} = \vxy{a_{x},a_{y}}$ und $\vv{b} = \vxy{b_{x},b_{y}}$ sind kollinear (parallel), wenn es eine Zahl $\lambda$ gibt, sodass
  \[
    \vv{a} = \lambda\cdot\vv{b} \qquad\Leftrightarrow\qquad \vxy{a_{x},a_{y}} = \lambda\cdot\vxy{b_{x},b_{y}}
  \]
\end{theorem}

\begin{center}
  \begin{tikzpicture}
    \tkzInit[xmin=0,xmax=12,ymin=0,ymax=4]
    \tkzAxeXY[]
    \tkzGrid[color=lightgray]

    \tkzDefPoint(1,3){A1}
    \tkzDefShiftPoint[A1](2,-1){A2}
    \tkzDrawSegment[red,thick,vector style](A1,A2)
    \tkzLabelSegment[red,above right](A1,A2){$\vv{a}$}

    \tkzDefPoint(9,1){B1}
    \tkzDefShiftPoint[B1](-6,3){B2}
    \tkzDrawSegment[red,thick,vector style](B1,B2)
    \tkzLabelSegment[red,above right](B1,B2){$\vv{b}$}

    \tkzDefPoint(3,0){C1}
    \tkzDefShiftPoint[C1](2,2){C2}
    \tkzDrawSegment[thick,vector style](C1,C2)
    \tkzLabelSegment[below right](C1,C2){$\vv{c}$}
  \end{tikzpicture}
\end{center}

\begin{example}
  \textbf{Beispiele:} Die Vektoren $\vv{a} = \vxy{2,-1}$ und $\vv{b} = \vxy{-6,3}$ sind kollinear, da:
  \[
     2 = \mathcolor{red}{-\frac{1}{3}\cdot} (-6) \qquad\qquad -1 = \mathcolor{red}{-\frac{1}{3}\cdot} 3
  \]

  Die Vektoren $\vv{a} = \vxy{2,-1}$ und $\vv{c} = \vxy{2,2}$ sind nicht parallel, da
  \[
    2 = \mathcolor{red}{1\cdot} 2 \qquad\qquad -1 = \mathcolor{red}{-\frac{1}{2}\cdot} 2
  \]
\end{example}

% ------------------------------------------------------------------------------
\subsection{Orthogonalität (senkrecht)}

Stehen zwei Vektoren senkrecht zueinander, so werden sie als \textbf{orthogonal} bezeichnet.

\begin{theorem}
  \textbf{Orthogonalität.} Zwei Vektoren $\vv{a} = \vxy{a_{x},a_{y}}$ und $\vv{b} = \vxy{b_{x},b_{y}}$ sind orthogonal (stehen senkrecht zueinander), wenn
  \[
    a_{x}\cdot b_{x}+a_{y}\cdot b_{y} = 0
  \]
\end{theorem}

\begin{center}
  \begin{tikzpicture}
    \tkzInit[xmin=0,xmax=12,ymin=0,ymax=4]
    \tkzAxeXY[]
    \tkzGrid[color=lightgray]

    \tkzDefPoint(1,3){A1}
    \tkzDefShiftPoint[A1](2,-3){A2}
    \tkzDrawSegment[red,thick,vector style](A1,A2)
    \tkzLabelSegment[red,below left](A1,A2){$\vv{a}$}

    \tkzDefPoint(3,0){B1}
    \tkzDefShiftPoint[B1](6,4){B2}
    \tkzDrawSegment[red,thick,vector style](B1,B2)
    \tkzLabelSegment[red,below right](B1,B2){$\vv{b}$}

    \tkzDefPoint(3,0){C1}
    \tkzDefShiftPoint[C1](1,4){C2}
    \tkzDrawSegment[thick,vector style](C1,C2)
    \tkzLabelSegment[below right](C1,C2){$\vv{c}$}

    \tkzMarkRightAngle[red,german,size=0.5](B2,B1,A1)
  \end{tikzpicture}
\end{center}

\begin{example}
  \textbf{Beispiele:} Die Vektoren $\vv{a} = \vxy{2,-3}$ und $\vv{b} = \vxy{6,4}$ sind orthogonal, da:
  \[
    2\cdot 6 + (-3)\cdot 4 = 12 - 12 = 0
  \]
  Die Vektoren $\vv{a} = \vxy{2,-3}$ und $\vv{c} = \vxy{1,4}$ sind nicht orthogonal, da
  \[
    2\cdot 1 + (-3)\cdot 4 = 2-12 = -10 \ne 0
  \]
\end{example}

Ist ein Vektor in Komponentenform gegeben, so können die zu ihm senkrechten Vektoren einfach ermittelt werden:

\begin{theorem}
  \textbf{Senkrechter Vektor.} Ist der Vektor $\vv{a} = \vxy{a_{x},a_{y}}$ gegeben, so stehen die Vektoren
  \[
    \vxy{-a_{y},a_{x}} \qquad\qquad \vxy{a_{y},-a_{x}}
  \]
  senkrecht auf $\vv{a}$ und haben die gleiche Länge wie $\vv{a}$. Die Komponenten der senkrechten Vektoren erhält man, indem die Komponenten von $\vv{a}$ vertauscht werden und eine Komponente negiert wird.
\end{theorem}
Das lässt sich einfach überprüfen:
\begin{align*}
  a_{x}\cdot(-a_{y})+a_{y}\cdot a_{x} = -a_{x}\cdot a_{y}+a_{y}\cdot a_{x} &= 0 \\
  a_{x}\cdot a_{y}+a_{y}\cdot(-a_{x}) = a_{x}\cdot a_{y}-a_{y}\cdot a_{x} &= 0
\end{align*}

\begin{center}
  \begin{tikzpicture}
    \tkzInit[xmin=0,xmax=10,ymin=0,ymax=5]
    \tkzAxeXY[]
    \tkzGrid[color=lightgray]

    \tkzDefPoint(5,3){A1}
    \tkzDefShiftPoint[A1](2,-3){A2}
    \tkzDrawSegment[thick,vector style](A1,A2)
    \tkzLabelSegment[below left](A1,A2){$\vv{a}$}

    \tkzDefShiftPoint[A1](3,2){B2}
    \tkzDrawSegment[red,thick,vector style](A1,B2)
    \tkzLabelSegment[red,above left](A1,B2){$\vv{b}$}

    \tkzDefShiftPoint[A1](-3,-2){C2}
    \tkzDrawSegment[red,thick,vector style](A1,C2)
    \tkzLabelSegment[red,above left](A1,C2){$-\vv{b}$}

    \tkzMarkRightAngle[red,german,size=0.5](C2,A1,A2)
  \end{tikzpicture}
\end{center}

\begin{example}
  \textbf{Beispiel:} Zum $\vv{a} = \vxy{2,-3}$ stehen die folgenden beiden Vektoren senkrecht:
  \[
    \vv{b} = \vxy{3,2} \qquad\qquad -\vv{b} = \vxy{-3,-2}
  \]
\end{example}
