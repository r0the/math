\newpage
\section{Vektoren und Punkte}

\subsection{Unterschiede}
Vektoren und Punkte sind unterschiedliche Objekte mit unterschiedlichen Eigenschaften.

Ein \textbf{Punkt} befindet sich an einem bestimmten Ort und er hat keine Richtung. Ein Punkt $P$ mit den \textbf{Koordinaten} $x$ und $y$ wird wie folgt geschrieben:
\[
  P(x\mid y)
\]

Ein \textbf{Vektor} kann beliebig verschoben werden, er hat also keine feste Position, aber er zeigt in eine bestimmte Richtung. Ein Vektor $\vv{v}$ mit den \textbf{Komponenten} $v_{x}$ und $v_{y}$ wird folgendermassen geschrieben:
\[
  \vv{v} = \vxy{v_{x}}{v_{y}}
\]

% ------------------------------------------------------------------------------
\subsection{Ortsvektor}

Wird ein Vektor $\vv{OP}$ mit dem Ursprung $O$ als Anfangspunkt und dem Punkt $P(x\mid y)$ als Endpunkt definiert wird, so entsprechen die Komponenten des Vektors gerade den Koordinaten des Punkts $P$:
\[
  \vv{OP} = \vxy{x-0}{y-0} = \vxy{x}{y}
\]
\begin{center}
  \begin{tikzpicture}
    \tkzInit[xmin=0,xmax=8,ymin=0,ymax=4]
    \tkzAxeXY[]
    \tkzGrid[color=lightgray]
    \tkzDefPoint(0,0){O}
    \tkzDefPoint(6,2){P}

    \tkzDrawSegment[red,thick,vector style](O,P)
    \tkzLabelSegment[red,above left](O,P){$\vv{OP}=\vxy{x}{y}$}

    \tkzDrawPoints(O,P)
    \tkzLabelPoint[above right](O){$O$}
    \tkzLabelPoint[above right](P){$P(x\mid y)$}
  \end{tikzpicture}
\end{center}

Dieser Vektor wird \textbf{Ortsvektor} des Punkts $P$ genannt. Der Ortsvektor von $P$ zeigt vom Ursprung $O$ zum Punkt $P$.

Werden Vektoren als Verschiebungen aufgefasst, so verschiebt der Ortsvektor den Ursprung $O$ in den Punkt $P$.
\begin{example}
  \textbf{Beispiel:} Der Ortsvektor des Punkts $A(-3\mid 5)$ lautet $\vv{OA} = \vxy{-3}{5}$.
\end{example}

% ------------------------------------------------------------------------------
\subsection{Vektor zwischen Punkten}

Ein Vektor kann durch einen Anfangspunkt $A$ und einen Endpunkt $B$ bestimmt werden.

\begin{center}
  \begin{tikzpicture}
    \tkzInit[xmin=0,xmax=12,ymin=0,ymax=4]
    \tkzAxeXY[]
    \tkzGrid[color=lightgray]
    \tkzDefPoint(0,0){O}
    \tkzDefPoint(2,3){A}
    \tkzDefPoint(9,1){B}

    \tkzDrawSegment[vector style](O,A)
    \tkzLabelSegment[above left](O,A){$\vv{OA}$}

    \tkzDrawSegment[vector style](O,B)
    \tkzLabelSegment[above left](O,B){$\vv{OB}$}

    \tkzDrawSegment[red,thick,vector style](A,B)
    \tkzLabelSegment[red,above right](A,B){$\vv{AB}$}

    \tkzDrawPoints(A,B)
    \tkzLabelPoint[left](A){$A(x_{A}\mid y_{A})$}
    \tkzLabelPoint[right](B){$B(x_{B}\mid y_{B})$}
  \end{tikzpicture}
\end{center}
Der Vektor $\vv{AB}$, welcher von Punkt $A$ nach Punkt $B$ zeigt, kann als Subtraktion der Ortsvektoren von $A$ und $B$ ausgedrückt werden.
\[
  \vv{AB} = \vv{OB} - \vv{OA} = \vxy{x_{B}}{y_{B}} - \vxy{x_{A}}{y_{A}} = \vxy{x_{B}-x_{A}}{y_{B}-y_{A}}
\]
Daraus lässt sich schliessen, dass sich die Komponenten des Vektors $\vv{AB}$ durch Subtraktion der Koordinaten von Punkt $B$ und Punkt $A$ ergeben.


\begin{theorem}
  \textbf{Vektor aus Punkten.} Hat ein Vektor $\vv{AB}$ den Anfangspunkt $A(x_{A}\mid y_{A})$ und den Endpunkt $B(x_{B}\mid y_{B})$ besitzt, so lautet seine Komponentendarstellung
  \[
    \vv{AB} = \vxy{x_{B}-x_{A}}{y_{B}-y_{A}}
  \]
\end{theorem}

\begin{example}
  \textbf{Beispiel:} Im der oben stehenden Abbildung ist der Vektor $\vv{v}$ durch den Anfangspunkt $A(2\mid 3)$ und den Endpunkt $B(9\mid 1)$ definiert. Also lautet seine Komponentendarstellung
  \[
    \vv{AB} = \vxy{9-2}{1-3} = \vxy{7}{-2}
  \]
\end{example}
