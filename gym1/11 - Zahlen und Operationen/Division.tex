\newpage
\section{Division und Brüche $a:b \quad \tfrac{a}{b}$}

So wie die Subtraktion die Gegenoperation zur Addition ist, so ist die Division die Gegenoperation zur Multiplikation.

% ----------------------------------------------------------------------------
\subsection{Definition}

In einem Korb sind immer fünf Äpfel. Haben wird drei solcher Körbe so haben wir insgesamt 15 Äpfel.
\[
  5\cdot 3 = 15
\]
Wenn wir hingegen zurück gehen und wissen möchten, wie 15 Äpfel gleichmässig auf drei
Körbe aufgeteilt werden, so rechnen wir umgekehrt:
\[
  15:3 = 5
\]
So ist das Gegenteil einer Multiplikation mit 3 gerade die Division durch 3.

\textbf{Definition:} Die Division ist die Umkehroperation der Multiplikation. Mit der Division wird ein unbekannter Faktor $x$ einer Multiplikation berechnet:
\[
  x \cdot b=a \qquad\Leftrightarrow\qquad x = a:b
\]

% ----------------------------------------------------------------------------
\subsection{Schreib- und Sprechweise}

Die Zahl $a$, welche dividiert wird, heisst \textbf{Dividend}. Die Zahl $b$, durch welche dividiert wird, heisst \textbf{Divisor}. Das Resultat einer Division wird \textbf{Quotient} genannt.

Für die Division wird das Operationszeichen $:$ verwendet. Wir schreiben
\[
  a : b = c
\]
und sagen «der Quotient von $a$ und $b$ ist gleich $c$.»

% ----------------------------------------------------------------------------
\subsection{Division durch Null}

Warum macht die Division durch Null Probleme? Nehmen wir eine beliebige Zahl $a$ und teilen diese durch $0$. Was soll das Ergebnis sein? Wir nennen das Ergebnis zunächst $x$ und schauen mal, ob wir $x$ herausfinden können.

Die Division durch Null muss speziell betrachtet werden.
\[
  x = a:0
\]
Dann müsste aber gelten, dass
\[
  0\cdot x = a
\]
ist. Die Zahl $a$ ist also doch nicht beliebig, sondern muss auch $0$ sein, sonst erhalten wir einen Widerspruch, denn auf der linken Seite der Gleichung steht $0\cdot x = 0$. Wenn $a = 0$ ist, erhalten wir
\[
  x = 0:0 \qquad\Leftrightarrow\qquad 0\cdot x = 0
\]
In die rechte Gleichung können wir für $x$ jede beliebige Zahl einsetzen und die Gleichung bleibt wahr. Jede beliebige Zahl ist aber kein sinnvolles Ergebnis für $x = 0 : 0$. Deshalb sagen wir, dass das Ergebnis einer Division durch Null \textbf{nicht definiert} ist.
