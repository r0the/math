\newpage
\section{Quadratwurzeln $\sqrt{a}$}

% ----------------------------------------------------------------------------
\subsection{Definition}

Welche Zahl muss quadriert werden, um Neun zu erhalten?
\[
  \square^{2} = 9
\]
Diese Frage wird mit Hilfe der Umkehroperation, dem Wurzelziehen oder Radizieren, beantwortet. Als Resultat ergibt sich
\[
  \sqrt{9} = \square
\]
Dabei ist zu beachten, dass es zwei Zahlen gibt, deren Quadrat Neun ist, nämlich $3$ und $-3$. Damit das Wurzelziehen eindeutig ist, wird festgelegt, dass das eine Wurzel nie negativ sein kann.

Welche Zahl muss quadriert werden, um $-9$ zu erhalten?
\[
  \square^{2}= -9
\]
Das ein Quadrat nie negativ sein kann, gibt es keine Antwort auf diese Frage. Das Wurzelziehen aus negativen Zahlen ist somit nicht definiert.


\textbf{Definition:} Das Wurzelziehen oder Radizieren ist die Umkehroperation des Quadrierens. Mit dem Radizieren wird die unbekannte, nicht-negative Basis eines Quadrates berechnet:
\begin{align*}
  x^{2} &= a \qquad\qquad x\ge 0\\
      x &= \sqrt{a} \qquad\qquad a\ge 0
\end{align*}
Das Radizieren einer negativen Zahl ist nicht definiert.

\begin{example}
  \textbf{Beispiele:} $\sqrt{25} = 5 \qquad \sqrt{0} = 0 \qquad \sqrt{1} = 1 \qquad \sqrt{-4}$ ist nicht definiert.
\end{example}


% ----------------------------------------------------------------------------
\subsection{Schreib- und Sprechweise}

Die Zahl $a$ welche radiziert wird, heisst \textbf{Radikand}. Das Resultat des Wurzelziehens heisst \textbf{Wurzel} oder \textbf{Radikal}.

Für das Wurzelziehen wird das Operationszeichen $\sqrt{\phantom{x}}$ verwendet. Wir schreiben
\[
  \sqrt{a} = b
\]
und sagen «Die Wurzel von $a$ ist gleich $b$.»

\begin{example}
  \textbf{Beispiel:} Die Wurzel von $49$ ist Sieben: $\sqrt{49} = 7$
\end{example}

% ----------------------------------------------------------------------------
\subsection{Gesetze}

Wurzeln können gemäss der folgenden Gesetze umgeformt werden.

\begin{theorem}
  \textbf{Produktregel.} Das Produkt zweier Wurzeln ist gleich der Wurzel aus dem Produkt der beiden Radikanden.
  \[
    \sqrt{a}\cdot\sqrt{b} = \sqrt{a\cdot b}
  \]
  Das Produkt zweier gleicher Wurzeln ist gleich dem Radikanden:
  \[
    \sqrt{a}\cdot\sqrt{a} = a
  \]
\end{theorem}

\begin{example}
  \textbf{Beispiele:} $\displaystyle \sqrt{3}\cdot\sqrt{5} = \sqrt{3\cdot 5} = \sqrt{15} \qquad\qquad \sqrt{2}\cdot\sqrt{2} = \sqrt{2\cdot 2} = \sqrt{4} = 2$
\end{example}

\begin{theorem}
  \textbf{Quotientenregel.} Der Quotient zweier Wurzeln ist gleich der Wurzel des Quotienten der beiden Radikanden.
  \[
    \frac{\sqrt{a}}{\sqrt{b}} = \sqrt{\frac{a}{b}}
  \]
\end{theorem}

\begin{example}
  \textbf{Beispiele:} $\displaystyle \frac{\sqrt{2}}{\sqrt{5}} = \sqrt{\frac{2}{5}} \qquad\qquad \frac{\sqrt{27}}{\sqrt{3}} = \sqrt{\frac{27}{3}} = \sqrt{9} = 3$
\end{example}

\begin{theorem}
  \textbf{Quadrieren einer Wurzel.} Das Quadrat einer Wurzel ist gleich dem Radikanden. Dabei muss natürlich beachtet werden, dass der Radikand nicht negativ sein kann.
  \[
    \left(\sqrt{a}\right)^{2} = a \qquad a\ge 0
  \]
\end{theorem}

\begin{example}
  \textbf{Beispiele:} $\displaystyle \left(\sqrt{5}\right)^{2} = 5 \qquad\qquad \sqrt{3}\cdot\sqrt{3} = \left(\sqrt{3}\right)^{2} = 3$
\end{example}

\begin{theorem}
  \textbf{Wurzel eines Quadrats.} Die Wurzel eines Quadrats ist gleich dem Betrag der Basis.
  \[
    \sqrt{a^{2}} = |a|
  \]
\end{theorem}
Hier kann $a$ durchaus eine negative Zahl sein, da ja der Radikand $a^{2}$ in jedem Fall positiv ist. Da das Resultat der Wurzel gemäss Definition nicht negativ sein kann, muss hier der Betrag von $a$ genommen werden.

\begin{example}
  \textbf{Beispiele:} $\displaystyle \sqrt{5^{2}} = |5| = 5 \qquad\qquad \sqrt{(-3)^{2}} = |-3| = 3$
\end{example}
