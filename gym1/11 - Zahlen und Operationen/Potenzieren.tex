\newpage
\section{Potenzieren $a^{b}$}

% ----------------------------------------------------------------------------
\subsection{Definition}

Das Potenzieren ist eine Rechenoperation der dritten Stufe. Die Potenz zweier natürlicher Zahlen entsteht durch das wiederholte Multiplizieren des gleichen Faktors $a$ mit dem Ausgangswert Eins:
\[
  a^{n} := 1\cdot\underbrace{a\cdot a\cdot a\cdots a}_{n-\text{mal}}
\]
Wenn in dieser Definition $n=0$ gesetzt wird, kommt $a$ gar nicht als Faktor vor und das Resultat ist $1$.
\begin{theorem}
  \textbf{Potenzieren mit Null:} Die nullte Potenz einer von Null verschiedenen Zahl $a$ ist gleich Eins.
  \[
    a^{0} = 1 \qquad a \ne 0
  \]
\end{theorem}

% ----------------------------------------------------------------------------
\subsection{Schreib- und Sprechweise}
Die Zahl $a$, welche potenziert wird, heisst \textbf{Basis}. Die Zahl $b$, um welche potenziert wird, heisst \textbf{Exponent}. Das Resultat einer Potenzierung heisst \textbf{Potenz}.

Für das Potenzieren wird kein Operationszeichen verwendet. Stattdessen wird der Exponent $b$ hochgestellt nach der Basis $a$ geschrieben:
\[
  a^{b} = c
\]
und sagen «$a$ hoch $b$ ist gleich $c$.» oder «Die $b$-te Potenz von $a$ ist gleich $c$.
\begin{example}
  \textbf{Beispiel:} Die vierte Potenz von Drei ist gleich $81$:
  \[
    3^{4} = 81
  \]
\end{example}
Das Potenzieren mit Zwei wird als \textbf{Quadrieren} bezeichnet, das Resultat dieser Operation als \textbf{Quadrat}.
\begin{example}
  \textbf{Beispiel:} Das Quadrat von Fünf ist gleich $25$:
  \[
    5^{2} = 25
  \]
\end{example}
\begin{note}
  \textbf{Achtung:} Für das Potenzieren gibt es weder ein Kommutativ- noch ein Assoziativgesetz.
  \[
    8 = 2^{3} \ne 3^{2} = 9 \qquad\qquad 8 = 2^{3} = \left(2^{1}\right)^{3} \ne 2^{(1^{3})} = 2^{1} = 2
  \]
\end{note}

\newpage
% ----------------------------------------------------------------------------
\subsection{Reihenfolge der Operationen}

Damit bei einer Rechenanweisung wie $2+3\cdot 5^{2}$ immer das gleiche Resultat entsteht, muss die Reihenfolge, in welcher die Operationen ausgeführt werden, klar geregelt werden. Dabei gelten folgende Vorschriften:

\textbf{Klammern zuerst.} Zuerst werden immer Operationen innerhalb von Klammern ausgeführt. Bei mehreren verschachtelten Klammern wird immer zuerst die innerste Klammer ausgerechnet.
\begin{example}
  \textbf{Beispiele:}
  \[
    5\cdot (2+3) = 5\cdot 5 = 25 \qquad\qquad (2\cdot 3)^{2} = 6^{2} = 36
  \]
\end{example}
\textbf{Dritte Stufe.} Danach werden Operationen der dritten Stufe ausgeführt, also Potenzieren und Wurzelziehen. Verschachtelte Potenzen werden von \textbf{oben nach unten} ausgeführt.
\begin{example}
  \textbf{Beispiele:}
  \[
    2\cdot 3^{2} = 2\cdot 9 = 18 \qquad\qquad 2^{3^{2}} = 2^{9} = 512 \qquad\qquad \left(2^{3}\right)^{2} = 8^{2} = 64
  \]
\end{example}
\begin{note}
  \textbf{Achtung:} Einfache Taschenrechner beachten diese Regel nicht und führen verschachtelte Potenzen von links nach rechts aus.
\end{note}
\textbf{Zweite Stufe (Punktoperationen).} Danach werden Operationen der zweiten Stufe ausgeführt, also Multiplikationen und Divisionen. Mehrere aufeinanderfolgende Operationen auf dieser Stufe werden von \textbf{links nach rechts} ausgeführt.
\begin{example}
  \textbf{Beispiele:}
  \[
    5\cdot 2+3 = 10+3 = 13 \qquad\qquad 2\cdot 3\cdot 4 = 6\cdot 4 = 24
  \]
\end{example}
\textbf{Erste Stufe (Strichoperationen).} Danach werden Operationen der ersten Stufe ausgeführt, also Additionen und Subtraktionen. Mehrere aufeinanderfolgende Operationen auf dieser Stufe werden von \textbf{links nach rechts} ausgeführt.
\begin{example}
  \textbf{Beispiele:}
  \[
    2+3+4 = 5+4 = 9 \qquad\qquad 2+(3+4) = 2+7 = 9
  \]
\end{example}
