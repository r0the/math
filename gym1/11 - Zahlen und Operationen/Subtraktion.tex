\newpage
\section{Subtraktion $a-b$}

% ----------------------------------------------------------------------------
\subsection{Definition}

Welche Zahl muss zu Fünf addiert werden, um Sieben zu erhalten?
\[
  5 + \square = 7
\]
Um diese Frage zu beantworten, muss die \textbf{Differenz} zwischen Fünf und Sieben bestimmt werden. Die Differenz zweier Zahlen wird mit Hilfe einer \textbf{Subtraktion} berechnet:
\[
  7 - 5 = \square
\]
\textbf{Definition:} Die Subtraktion ist die Umkehroperation der Addition. Mit der Subtraktion wird der unbekannte Summand $x$ einer Addition berechnet:

\begin{align*}
  a+x &= c \\
    x &= c-a
\end{align*}

% ----------------------------------------------------------------------------
\subsection{Schreib- und Sprechweise}

Die Zahl $a$, von welcher subtrahiert wird, heisst \textbf{Minuend}. Die Zahl $b$, welche subtrahiert wird, heisst \textbf{Subtrahend}. Das Resultat einer Subtraktion wird \textbf{Differenz} genannt.

Für die Subtraktion wird das Operationszeichen $-$ verwendet. Wir schreiben
\[
  a - b = c
\]
und sagen «die Differenz von $a$ und $b$ ist gleich $c$.»

% ----------------------------------------------------------------------------
\subsection{Reihenfolge der Operationen}

Die Subtraktion ist eine Operation der ersten Stufe. Das bedeutet, dass Subtraktionen immer gleichzeitig mit den Additionen von links nach rechts ausgeführt werden.

\begin{example}
  \textbf{Beispiele:}
  \begin{align*}
      3-2+1 &= 1+1 = 2 &   5-2\cdot 2 &= 5-4 = 1 \\
    3-(2+1) &= 3-3 = 0 & (5-2)\cdot 2 &= 3\cdot 2 = 6
  \end{align*}
\end{example}

\begin{note}
  \textbf{Achtung:} Für die Subtraktion gibt es weder ein Kommutativ- noch ein Assoziativgesetz.
  \[
    -1 = 2-3 \ne 3-2 = 1 \qquad\qquad 0 = 1-1 = 3-2-1 \ne 3-(2-1) = 3-1 = 2
  \]
\end{note}
