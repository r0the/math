\newpage
\section{Grundlegende Umformungen}

Im Thema «Zahlen und Operationen» haben wir schon viele Gesetze kennengelernt, nach welchen Terme umgeformt werden können. Diese Gesetze werden hier nochmals zusammengefasst.

Sämtliche Gesetze gelten auch, wenn die Variablen durch Terme ersetzt werden.

% ----------------------------------------------------------------------------
\subsection{Addition und Multiplikation}

\textbf{Kommutativgesetz:} Die Addition und Multiplikation sind kommutativ. Das bedeutet, dass die Summanden bzw. Faktoren vertauscht werden können:
\[
  a+b = b+a \qquad\qquad a\cdot b = b\cdot a
\]

\textbf{Assoziativgesetz:} Die Addition und Multiplikation sind assoziativ. Das bedeutet, dass es keine Rolle spielt, ob bei mehreren gleichen Operationen die linke oder rechte zuerst ausgeführt wird.
\[
  (a + b) + c = a + (b + c) \qquad\qquad (a\cdot b)\cdot c = a\cdot(b\cdot c)
\]


\textbf{Neutralität:} Null ist neutral bezüglich der Addition. Wird Null zu einer beliebigen Zahl addiert, ergibt sich wieder diese Zahl. Analog ist die Eins neutral bezüglich der Multiplikation.
\[
  a + 0 = 0 + a = a \qquad\qquad a\cdot 1 = 1\cdot a = a
\]

\textbf{Distributivgesetz.} Eine Summe wird mit einem Faktor $a$ multipliziert, indem jeder Summand mit $a$ multipliziert wird.
\[
  a \cdot (b + c) = a \cdot b + a \cdot c
\]

\textbf{Multiplikation mit Null:} Wird eine beliebige Zahl mit Null multipliziert, so ergibt sich immer Null:
\[
  a\cdot 0 = 0\cdot a = 0
\]

% ----------------------------------------------------------------------------
\subsection{Subtraktion und Gegenzahlen}

\textbf{Gegen-Gegenzahl.} Die Gegenzahl der Gegenzahl einer Zahl ist die ursprüngliche Zahl:
\[
  -(-a) = a
\]

\textbf{Addition und Subtraktion.} Das Addieren der Gegenzahl ist das Gleiche wie das Subtrahieren der Zahl. Das Subtrahieren der Gegenzahl ist das gleiche wie das Addieren der Zahl.
\[
  a+(-b) = a-b \qquad\qquad a-(-b) = a+b
\]

\textbf{Subtraktion einer Summe.} Die Subtraktion einer Summe ist das Gleiche wie das Subtrahieren beider Summanden:
\[
  a-(b+c) = a-b-c
\]

\textbf{Subtraktion einer Differenz.} Die Subtraktion einer Differenz ist das Gleiche wie das Subtrahieren des Minuenden und das addieren des Subtrahenden.
\[
  a-(b-c) = a-b+c
\]

\textbf{Multiplikation von Gegenzahlen.} Für die Multiplikation von Gegenzahlen gelten folgende Regeln:
\[
  (-a)\cdot b = a\cdot(-b) = -ab \qquad\qquad  (-a)\cdot(-b) = ab
\]

% ----------------------------------------------------------------------------
\subsection{Potenzen}

\textbf{Negative Basis.} Die Potenz einer negativen Zahl $-n$ mit einem geraden Exponenten $b$ ist gleich der Potenz mit der Gegenzahl $n$ als Basis.
  \[
    (-n)^{b} = n^{b} \qquad \text{wenn}\;b\;\text{gerade}
  \]
  Die Potenz einer negativen Zahl $-n$ mit einem ungeraden Exponenten $b$ ist gleich der Gegenzahl der Potenz mit der Gegenzahl $n$ als Basis.
  \[
    (-n)^{b} = -n^{b} \qquad \text{wenn}\;b\;\text{ungerade}
  \]

\textbf{Produkt mit gleicher Basis.} Potenzen mit gleicher Basis werden multipliziert, indem die Potenzen addiert werden:
\[
  a^{k} \cdot a^{m} = a^{k+m}
\]

\textbf{Quotient mit gleicher Basis.} Potenzen mit gleicher Basis werden dividiert, indem die Potenzen subtrahiert werden:
\[
  \frac{a^{k}}{a^{m}} = a^{k-m}
\]

\textbf{Produkt mit gleichem Exponent.} In einem Produkt können gleiche Exponenten ausgeklammert werden:
\[
  a^{k}\cdot b^{k} = (a\cdot b)^{k}
\]

\textbf{Quotient mit gleichem Exponent.} In einem Bruch können gleiche Exponenten ausgeklammert werden:
\[
  \frac{a^{k}}{b^{k}} = \left(\frac{a}{b}\right)^{k}
\]

\textbf{Potenz einer Potenz.} Potenzen werden potenziert, indem die Exponenten multipliziert werden:
\[
  \left(a^{k}\right)^{m} = \left(a^{m}\right)^{k}= a^{k\cdot m}
\]

  % ----------------------------------------------------------------------------
\subsection{Quadratwurzeln}

\textbf{Produktregel.} Das Produkt zweier Wurzeln ist gleich der Wurzel aus dem Produkt der beiden Radikanden.
\[
  \sqrt{a}\cdot\sqrt{b} = \sqrt{a\cdot b}
\]
Das Produkt zweier gleicher Wurzeln ist gleich dem Radikanden:
\[
  \sqrt{a}\cdot\sqrt{a} = a
\]

\textbf{Quotientenregel.} Der Quotient zweier Wurzeln ist gleich der Wurzel des Quotienten der beiden Radikanden.
\[
  \frac{\sqrt{a}}{\sqrt{b}} = \sqrt{\frac{a}{b}}
\]

\textbf{Quadrieren einer Wurzel.} Das Quadrat einer Wurzel ist gleich dem Radikanden. Dabei muss natürlich beachtet werden, dass der Radikand nicht negativ sein kann.
\[
  \left(\sqrt{a}\right)^{2} = a \qquad a\ge 0
\]

\textbf{Wurzel eines Quadrats.} Die Wurzel eines Quadrats ist gleich dem Betrag der Basis.
\[
  \sqrt{a^{2}} = |a|
\]
