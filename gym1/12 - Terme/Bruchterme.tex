\newpage
\section{Bruchterme}

% ----------------------------------------------------------------------------
\subsection{Definitionsmenge}
Wenn bei einem Term eine Variable im Nenner vorkommt, so muss durch Angabe der entsprechenden Definitionsmenge sichergestellt werden, dass der Nenner nicht Null werden kann. Dies geschieht am einfachsten, indem die «gefährlichen» Zahlen aus der Grundmenge (normalerweise die reellen Zahlen) ausgeschlossen werden.

Was im Zähler steht, spielt hierbei keine Rolle.
\begin{example}
  \textbf{Beispiele:}
  \begin{align*}
    \frac{1}{a} &\qquad\Rightarrow\qquad \mathbb{D}_{a} = \mathbb{R} \setminus \{0\} &
    \frac{x^{2}}{x-2} &\qquad\Rightarrow\qquad \mathbb{D}_{x} = \mathbb{R} \setminus \{2\} \\[4mm]
    \frac{5z+2}{(z+1)^{2}} &\qquad\Rightarrow\qquad \mathbb{D}_{z} = \mathbb{R} \setminus \{-1\}
  \end{align*}
\end{example}

% ----------------------------------------------------------------------------
\subsection{Kürzen}
Ein Bruchterm darf nur gekürzt werden, wenn sowohl Zähler als auch Nenner als \textbf{Produkt} vorliegen. Wenn in Zähler oder Nenner eine Strichoperation vorkommt, welche nicht in Klammern steht, so ist dies eine Summe und nicht ein Produkt.
\begin{example}
  \textbf{Beispiel:}
  \begin{align*}
    \frac{a+2}{a+3} && \frac{a\cdot 2}{a\cdot 3} = \frac{\cancel{a}\cdot 2}{\cancel{a}\cdot 3}= \frac{2}{3}
  \end{align*}
  Links liegen Summen vor, es darf nicht gekürzt werden, rechts liegen Produkte vor, es kann mit dem Faktor $a$ gekürzt werden.
\end{example}

Um einen Bruchterm zu kürzen, müssen also zunächst Zähler und Nenner in Produkte umgewandelt werden. Dies nennen wir auch \textbf{Faktorisieren}, dazu gibt es ein separates Merkblatt.

\begin{example}
  \textbf{Beispiel:}
  \begin{align*}
    \frac{5a+5}{a^{2}+a} = \frac{5\cdot (a+1)}{a\cdot (a+1)} = \frac{5\cdot \cancel{(a+1)}}{a\cdot \cancel{(a+1)}} = \frac{5}{a}
  \end{align*}
  Hier werden zunächst Zähler und Nenner mit Hilfe des Distributivgesetzes faktorisiert. Anschliessend kann mit dem Faktor $(a+1)$ gekürzt werden.
\end{example}

% ----------------------------------------------------------------------------
\subsection{Multiplizieren}
Bruchterme werden wie Brüche multipliziert, indem die Zähler und Nenner multipliziert werden. Wenn in einem Zähler oder Nenner eine Summe steht, müssen unbedingt Klammern gesetzt werden.
\begin{example}
  \textbf{Beispiel:}
  \begin{align*}
    \frac{x+y}{x-y}\cdot \frac{x}{y} = \frac{(x+y)\cdot x}{(x-y)\cdot y} = \frac{x^{2}+xy}{xy-y^{2}}
  \end{align*}
  Hier werden im letzten Schritt Zähler und Nenner ausmultipliziert. Dies sollte immer nur zuletzt gemacht werden. Wenn noch weiter umgeformt wird, ist es besser, Zähler und Nenner als Produkt beizubehalten.
\end{example}

% ----------------------------------------------------------------------------
\subsection{Addieren}
Bruchterme werden ebenfalls wie Brüche addiert, indem die Nenner gleichnamig gemacht und dann die Zähler addiert werden.
\begin{example}
  \[
    \frac{2}{1-x}+ \frac{3}{x} = \frac{2\cdot x}{(1-x)\cdot x}+\frac{3\cdot (1-x)}{x\cdot (1-x)} = \frac{2x+3-3x}{(1-x)x} = \frac{5x+3}{(1-x)x}
  \]
\end{example}
