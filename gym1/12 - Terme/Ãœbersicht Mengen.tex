\documentclass[parskip=half]{scrartcl}
\input{../../../ros-cheatsheet.tex}

\title{Mengen}
\id{12.81}
\author{Stefan Rothe}
\date{29.04.2024}

\begin{document}
  \maketitle

  \subsection*{Schreibweise}
  \begin{itemize}
    \item \textbf{Definition durch Aufzählung}: Die Menge der genügenden Noten ist $G = \{ 4; 5; 6 \}$
    \item \textbf{Definition durch Bedingung}: Die Menge der ungenügenden Noten sind die natürlichen Zahlen zwischen Eins und Drei: $U = \{ n\in \mathbb{N} \mid 1 \le n \le 3 \}$
    \item \textbf{Element}: Vier ist eine genügende Note: $4\in G$.
    \item \textbf{Vereinigung}: alle Noten sind $M = G\cup U$.
    \item \textbf{Subtraktion}: alle Noten ohne die Eins: $M \setminus \{1\}$.

  \end{itemize}

  \subsection*{Zahlenmengen}
  \begin{center}
    \begin{tabularx}{0.9\textwidth}{lX}
    \toprule
      natürliche Zahlen & $\mathbb{N} = \{ 0, 1, 2, 3, \ldots \}$ \\
    \midrule
      ganze Zahlen & $\mathbb{Z} = \{\ldots, -3, -2, -1, 0, 1, 2, \ldots \}$ \\
    \midrule
      rationale Zahlen & $\mathbb{Q} = \left\{\frac{z}{n}\quad z\in\mathbb{Z}, n\in \mathbb{N}^{+}\right\}$ \\
    \midrule
      irrationale Zahlen & $\mathbb{I} = \{ \pi, \sqrt{2}, \ldots \}$ \\
    \midrule
      reelle Zahlen & $\mathbb{R} = \mathbb{Q}\cup \mathbb{I}$ \\
    \bottomrule
    \end{tabularx}
  \end{center}

  \subsection*{Teilmengen}
  \begin{itemize}
    \item positive Zahlen: $\mathbb{N}^{+}, \quad \mathbb{Q}^{+},  \quad \mathbb{R}^{+}$
    \item negative Zahlen: $\mathbb{N}^{-}, \quad \mathbb{Q}^{-},  \quad \mathbb{R}^{-}$
    \item nicht-negative Zahlen: $\mathbb{N}_{0}^{+}, \quad \mathbb{Q}_{0}^{+}, \quad \mathbb{R}_{0}^{+}$
    \item nicht-positive Zahlen: $\mathbb{N}_{0}^{-}, \quad \mathbb{Q}_{0}^{-}, \quad \mathbb{R}_{0}^{-}$
  \end{itemize}

  \subsection*{Intervalle}
  \begin{center}
    \def\arraystretch{1.1}
    \begin{tabularx}{0.9\textwidth}{XXX}
    \toprule
      geschlossenes Intervall & $[a, b]$ & $\{ x\in\mathbb{R} \;|\; a \leq x \leq b\}$ \\
    \midrule
      linksoffenes Intervall & $(a, b]$ oder $]a, b]$ & $\{ x\in\mathbb{R} \;|\; a< x \leq b\}$ \\
    \midrule
      rechtsoffenes Intervall & $[a, b)$ oder $[a, b[$ & $\{ x\in\mathbb{R} \;|\; a \leq x < b\}$ \\
    \midrule
      offenes Intervall & $(a, b)$ oder $]a, b[$ & $\{ x\in\mathbb{R} \;|\; a< x < b\}$ \\
    \bottomrule
    \end{tabularx}
  \end{center}

\end{document}
